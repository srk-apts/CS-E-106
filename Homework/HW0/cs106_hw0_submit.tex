\documentclass[]{article}
\usepackage{lmodern}
\usepackage{amssymb,amsmath}
\usepackage{ifxetex,ifluatex}
\usepackage{fixltx2e} % provides \textsubscript
\ifnum 0\ifxetex 1\fi\ifluatex 1\fi=0 % if pdftex
  \usepackage[T1]{fontenc}
  \usepackage[utf8]{inputenc}
\else % if luatex or xelatex
  \ifxetex
    \usepackage{mathspec}
  \else
    \usepackage{fontspec}
  \fi
  \defaultfontfeatures{Ligatures=TeX,Scale=MatchLowercase}
\fi
% use upquote if available, for straight quotes in verbatim environments
\IfFileExists{upquote.sty}{\usepackage{upquote}}{}
% use microtype if available
\IfFileExists{microtype.sty}{%
\usepackage{microtype}
\UseMicrotypeSet[protrusion]{basicmath} % disable protrusion for tt fonts
}{}
\usepackage[margin=1in]{geometry}
\usepackage{hyperref}
\hypersetup{unicode=true,
            pdftitle={CS-E-106: Data Modeling},
            pdfauthor={Instructor: Hakan Gogtas; Submitted by: Saurabh Kulkarni},
            pdfborder={0 0 0},
            breaklinks=true}
\urlstyle{same}  % don't use monospace font for urls
\usepackage{color}
\usepackage{fancyvrb}
\newcommand{\VerbBar}{|}
\newcommand{\VERB}{\Verb[commandchars=\\\{\}]}
\DefineVerbatimEnvironment{Highlighting}{Verbatim}{commandchars=\\\{\}}
% Add ',fontsize=\small' for more characters per line
\usepackage{framed}
\definecolor{shadecolor}{RGB}{248,248,248}
\newenvironment{Shaded}{\begin{snugshade}}{\end{snugshade}}
\newcommand{\AlertTok}[1]{\textcolor[rgb]{0.94,0.16,0.16}{#1}}
\newcommand{\AnnotationTok}[1]{\textcolor[rgb]{0.56,0.35,0.01}{\textbf{\textit{#1}}}}
\newcommand{\AttributeTok}[1]{\textcolor[rgb]{0.77,0.63,0.00}{#1}}
\newcommand{\BaseNTok}[1]{\textcolor[rgb]{0.00,0.00,0.81}{#1}}
\newcommand{\BuiltInTok}[1]{#1}
\newcommand{\CharTok}[1]{\textcolor[rgb]{0.31,0.60,0.02}{#1}}
\newcommand{\CommentTok}[1]{\textcolor[rgb]{0.56,0.35,0.01}{\textit{#1}}}
\newcommand{\CommentVarTok}[1]{\textcolor[rgb]{0.56,0.35,0.01}{\textbf{\textit{#1}}}}
\newcommand{\ConstantTok}[1]{\textcolor[rgb]{0.00,0.00,0.00}{#1}}
\newcommand{\ControlFlowTok}[1]{\textcolor[rgb]{0.13,0.29,0.53}{\textbf{#1}}}
\newcommand{\DataTypeTok}[1]{\textcolor[rgb]{0.13,0.29,0.53}{#1}}
\newcommand{\DecValTok}[1]{\textcolor[rgb]{0.00,0.00,0.81}{#1}}
\newcommand{\DocumentationTok}[1]{\textcolor[rgb]{0.56,0.35,0.01}{\textbf{\textit{#1}}}}
\newcommand{\ErrorTok}[1]{\textcolor[rgb]{0.64,0.00,0.00}{\textbf{#1}}}
\newcommand{\ExtensionTok}[1]{#1}
\newcommand{\FloatTok}[1]{\textcolor[rgb]{0.00,0.00,0.81}{#1}}
\newcommand{\FunctionTok}[1]{\textcolor[rgb]{0.00,0.00,0.00}{#1}}
\newcommand{\ImportTok}[1]{#1}
\newcommand{\InformationTok}[1]{\textcolor[rgb]{0.56,0.35,0.01}{\textbf{\textit{#1}}}}
\newcommand{\KeywordTok}[1]{\textcolor[rgb]{0.13,0.29,0.53}{\textbf{#1}}}
\newcommand{\NormalTok}[1]{#1}
\newcommand{\OperatorTok}[1]{\textcolor[rgb]{0.81,0.36,0.00}{\textbf{#1}}}
\newcommand{\OtherTok}[1]{\textcolor[rgb]{0.56,0.35,0.01}{#1}}
\newcommand{\PreprocessorTok}[1]{\textcolor[rgb]{0.56,0.35,0.01}{\textit{#1}}}
\newcommand{\RegionMarkerTok}[1]{#1}
\newcommand{\SpecialCharTok}[1]{\textcolor[rgb]{0.00,0.00,0.00}{#1}}
\newcommand{\SpecialStringTok}[1]{\textcolor[rgb]{0.31,0.60,0.02}{#1}}
\newcommand{\StringTok}[1]{\textcolor[rgb]{0.31,0.60,0.02}{#1}}
\newcommand{\VariableTok}[1]{\textcolor[rgb]{0.00,0.00,0.00}{#1}}
\newcommand{\VerbatimStringTok}[1]{\textcolor[rgb]{0.31,0.60,0.02}{#1}}
\newcommand{\WarningTok}[1]{\textcolor[rgb]{0.56,0.35,0.01}{\textbf{\textit{#1}}}}
\usepackage{graphicx,grffile}
\makeatletter
\def\maxwidth{\ifdim\Gin@nat@width>\linewidth\linewidth\else\Gin@nat@width\fi}
\def\maxheight{\ifdim\Gin@nat@height>\textheight\textheight\else\Gin@nat@height\fi}
\makeatother
% Scale images if necessary, so that they will not overflow the page
% margins by default, and it is still possible to overwrite the defaults
% using explicit options in \includegraphics[width, height, ...]{}
\setkeys{Gin}{width=\maxwidth,height=\maxheight,keepaspectratio}
\IfFileExists{parskip.sty}{%
\usepackage{parskip}
}{% else
\setlength{\parindent}{0pt}
\setlength{\parskip}{6pt plus 2pt minus 1pt}
}
\setlength{\emergencystretch}{3em}  % prevent overfull lines
\providecommand{\tightlist}{%
  \setlength{\itemsep}{0pt}\setlength{\parskip}{0pt}}
\setcounter{secnumdepth}{0}
% Redefines (sub)paragraphs to behave more like sections
\ifx\paragraph\undefined\else
\let\oldparagraph\paragraph
\renewcommand{\paragraph}[1]{\oldparagraph{#1}\mbox{}}
\fi
\ifx\subparagraph\undefined\else
\let\oldsubparagraph\subparagraph
\renewcommand{\subparagraph}[1]{\oldsubparagraph{#1}\mbox{}}
\fi

%%% Use protect on footnotes to avoid problems with footnotes in titles
\let\rmarkdownfootnote\footnote%
\def\footnote{\protect\rmarkdownfootnote}

%%% Change title format to be more compact
\usepackage{titling}

% Create subtitle command for use in maketitle
\providecommand{\subtitle}[1]{
  \posttitle{
    \begin{center}\large#1\end{center}
    }
}

\setlength{\droptitle}{-2em}

  \title{\textbf{CS-E-106: Data Modeling}}
    \pretitle{\vspace{\droptitle}\centering\huge}
  \posttitle{\par}
  \subtitle{\textbf{Assignment 0}}
  \author{\textbf{Instructor: Hakan Gogtas} \\ \textbf{Submitted by:} Saurabh Kulkarni}
    \preauthor{\centering\large\emph}
  \postauthor{\par}
      \predate{\centering\large\emph}
  \postdate{\par}
    \date{\textbf{Due Date:} 09/16/2019}


\begin{document}
\maketitle

\textbf{Solution 1:}

\textbf{(a).}

\[f_X(x) = ax^{a-1}\]

\[\mathbb{E}[X] = \int_{-\inf}^{\inf}x.f_X(x)dx = \int_{0}^{1}x.a.x^{a-1}dx = a\int_{0}^{1}x^{a}dx = a\frac{1}{(a+1)}x^{a+1}|_{0}^{1} = \frac{a}{a+1}\]

\[Var[X] = \mathbb{E}[X^{2}]-(\mathbb{E}[X])^{2} = \int_{0}^{1}x^{2}.a.x^{a-1}dx-(\mathbb{E}[X])^{2} = 
a\int_{0}^{1}x^{a+1}dx-(\mathbb{E}[X])^{2} = a\frac{1}{(a+2)}x^{a+2}|_{0}^{1}-(\frac{a}{a+1})^{2}=\frac{a}{(a+2)(a+1)^{2}}\]

\textbf{(b).} \[f_X(x) = \frac{x}{n}\]

\[\mathbb{E}[X] = \sum_{-\inf}^{\inf}xf_X(x) = \frac{1}{n}\sum_{x=1}^{n}x^{2} = \frac{(n+1)(2n+1)}{6}\]

\[Var(X) = \mathbb{E}[X^{2}]-(\mathbb{E}[X])^{2} = \frac{1}{n}\sum_{x=1}^{n}x^{3}-(\mathbb{E}[X])^{2} = (\frac{n(n+1)}{2})^{2}-(\frac{(n+1)(2n+1)}{6})^{2} = -\frac{(n+1)(4n^{2})-5n+1)}{36}\]

\textbf{(c).} \[f_X(x) = \frac{3}{2}(x-1)^{2}\]

\[\mathbb{E}[X] = \int_{-\inf}^{\inf}x.f_X(x)dx = \int_{0}^{2}x.\frac{3}{2}(x-1)^{2}dx = \frac{3}{2}\int_{0}^{2}(x^{3}-2x^{2}+x)dx = \frac{3}{2}[\frac{1}{4}x^{4}-\frac{2}{3}x^{3}+\frac{1}{2}x^{2}]|_0^2=1\]

\[Var(X) = \mathbb{E}[X^{2}]-(\mathbb{E}[X])^{2} = \int_{0}^{2}x^{2}.\frac{3}{2}(x-1)^{2}dx-1 = \frac{3}{2}[\frac{1}{5}x^{5}-\frac{1}{2}x^{4}+\frac{1}{3}x^{3}]|_0^2-1=\frac{3}{5}\]

\textbf{R Programming Questions}

\begin{Shaded}
\begin{Highlighting}[]
\NormalTok{X <-}\StringTok{ }\KeywordTok{matrix}\NormalTok{(}\KeywordTok{c}\NormalTok{(}\DecValTok{10}\NormalTok{,}\DecValTok{3}\NormalTok{,}\DecValTok{5}\NormalTok{,}\DecValTok{1}\NormalTok{,}\DecValTok{8}\NormalTok{,}\DecValTok{2}\NormalTok{,}\DecValTok{9}\NormalTok{,}\DecValTok{7}\NormalTok{,}\DecValTok{4}\NormalTok{), }\DataTypeTok{nrow =} \DecValTok{3}\NormalTok{, }\DataTypeTok{ncol =} \DecValTok{3}\NormalTok{)}
\KeywordTok{print}\NormalTok{(X)}
\end{Highlighting}
\end{Shaded}

\begin{verbatim}
##      [,1] [,2] [,3]
## [1,]   10    1    9
## [2,]    3    8    7
## [3,]    5    2    4
\end{verbatim}

\begin{Shaded}
\begin{Highlighting}[]
\NormalTok{Y <-}\StringTok{ }\KeywordTok{matrix}\NormalTok{(}\KeywordTok{c}\NormalTok{(}\DecValTok{2}\NormalTok{,}\DecValTok{8}\NormalTok{,}\DecValTok{3}\NormalTok{,}\DecValTok{5}\NormalTok{,}\DecValTok{1}\NormalTok{,}\DecValTok{12}\NormalTok{,}\DecValTok{13}\NormalTok{,}\DecValTok{4}\NormalTok{,}\DecValTok{7}\NormalTok{), }\DataTypeTok{nrow =} \DecValTok{3}\NormalTok{, }\DataTypeTok{ncol =} \DecValTok{3}\NormalTok{)}
\KeywordTok{print}\NormalTok{(Y)}
\end{Highlighting}
\end{Shaded}

\begin{verbatim}
##      [,1] [,2] [,3]
## [1,]    2    5   13
## [2,]    8    1    4
## [3,]    3   12    7
\end{verbatim}

\textbf{Solution 6:}

\begin{Shaded}
\begin{Highlighting}[]
\KeywordTok{print}\NormalTok{(X}\OperatorTok{+}\NormalTok{Y)}
\end{Highlighting}
\end{Shaded}

\begin{verbatim}
##      [,1] [,2] [,3]
## [1,]   12    6   22
## [2,]   11    9   11
## [3,]    8   14   11
\end{verbatim}

\textbf{Solution 7:}

\begin{Shaded}
\begin{Highlighting}[]
\KeywordTok{solve}\NormalTok{((}\KeywordTok{t}\NormalTok{(X)}\OperatorTok\NormalTok{X))}\OperatorTok\KeywordTok{t}\NormalTok{(X)}\OperatorTok\NormalTok{Y}
\end{Highlighting}
\end{Shaded}

\begin{verbatim}
##            [,1]      [,2]       [,3]
## [1,]  0.4563107  6.563107  1.6019417
## [2,]  1.1941748  3.941748  0.2135922
## [3,] -0.4174757 -7.174757 -0.3592233
\end{verbatim}

\textbf{Solution 8:}

\begin{Shaded}
\begin{Highlighting}[]
\NormalTok{samples <-}\StringTok{ }\KeywordTok{runif}\NormalTok{(}\DecValTok{10000}\NormalTok{)}
\NormalTok{percentile <-}\StringTok{ }\KeywordTok{quantile}\NormalTok{(samples, }\DataTypeTok{probs =} \KeywordTok{c}\NormalTok{(}\FloatTok{0.005}\NormalTok{,}\FloatTok{0.995}\NormalTok{))}
\KeywordTok{print}\NormalTok{(percentile)}
\end{Highlighting}
\end{Shaded}

\begin{verbatim}
##        0.5%       99.5% 
## 0.006170633 0.995056324
\end{verbatim}


\end{document}
